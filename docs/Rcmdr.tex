% Options for packages loaded elsewhere
\PassOptionsToPackage{unicode}{hyperref}
\PassOptionsToPackage{hyphens}{url}
%
\documentclass[
]{book}
\title{설치형 오픈 통계 패키지 - \texttt{Rcmdr}}
\author{신종화, 이광춘, 유충현, 홍성학}
\date{2022-04-15}

\usepackage{amsmath,amssymb}
\usepackage{lmodern}
\usepackage{iftex}
\ifPDFTeX
  \usepackage[T1]{fontenc}
  \usepackage[utf8]{inputenc}
  \usepackage{textcomp} % provide euro and other symbols
\else % if luatex or xetex
  \usepackage{unicode-math}
  \defaultfontfeatures{Scale=MatchLowercase}
  \defaultfontfeatures[\rmfamily]{Ligatures=TeX,Scale=1}
\fi
% Use upquote if available, for straight quotes in verbatim environments
\IfFileExists{upquote.sty}{\usepackage{upquote}}{}
\IfFileExists{microtype.sty}{% use microtype if available
  \usepackage[]{microtype}
  \UseMicrotypeSet[protrusion]{basicmath} % disable protrusion for tt fonts
}{}
\makeatletter
\@ifundefined{KOMAClassName}{% if non-KOMA class
  \IfFileExists{parskip.sty}{%
    \usepackage{parskip}
  }{% else
    \setlength{\parindent}{0pt}
    \setlength{\parskip}{6pt plus 2pt minus 1pt}}
}{% if KOMA class
  \KOMAoptions{parskip=half}}
\makeatother
\usepackage{xcolor}
\IfFileExists{xurl.sty}{\usepackage{xurl}}{} % add URL line breaks if available
\IfFileExists{bookmark.sty}{\usepackage{bookmark}}{\usepackage{hyperref}}
\hypersetup{
  pdftitle={설치형 오픈 통계 패키지 - Rcmdr},
  pdfauthor={신종화, 이광춘, 유충현, 홍성학},
  hidelinks,
  pdfcreator={LaTeX via pandoc}}
\urlstyle{same} % disable monospaced font for URLs
\usepackage{longtable,booktabs,array}
\usepackage{calc} % for calculating minipage widths
% Correct order of tables after \paragraph or \subparagraph
\usepackage{etoolbox}
\makeatletter
\patchcmd\longtable{\par}{\if@noskipsec\mbox{}\fi\par}{}{}
\makeatother
% Allow footnotes in longtable head/foot
\IfFileExists{footnotehyper.sty}{\usepackage{footnotehyper}}{\usepackage{footnote}}
\makesavenoteenv{longtable}
\usepackage{graphicx}
\makeatletter
\def\maxwidth{\ifdim\Gin@nat@width>\linewidth\linewidth\else\Gin@nat@width\fi}
\def\maxheight{\ifdim\Gin@nat@height>\textheight\textheight\else\Gin@nat@height\fi}
\makeatother
% Scale images if necessary, so that they will not overflow the page
% margins by default, and it is still possible to overwrite the defaults
% using explicit options in \includegraphics[width, height, ...]{}
\setkeys{Gin}{width=\maxwidth,height=\maxheight,keepaspectratio}
% Set default figure placement to htbp
\makeatletter
\def\fps@figure{htbp}
\makeatother
\setlength{\emergencystretch}{3em} % prevent overfull lines
\providecommand{\tightlist}{%
  \setlength{\itemsep}{0pt}\setlength{\parskip}{0pt}}
\setcounter{secnumdepth}{5}
\usepackage{booktabs}
\usepackage{amsthm}
\makeatletter
\def\thm@space@setup{%
  \thm@preskip=8pt plus 2pt minus 4pt
  \thm@postskip=\thm@preskip
}
\makeatother
\ifLuaTeX
  \usepackage{selnolig}  % disable illegal ligatures
\fi
\usepackage[]{natbib}
\bibliographystyle{apalike}

\begin{document}
\maketitle

{
\setcounter{tocdepth}{1}
\tableofcontents
}
\hypertarget{uxb4e4uxc5b4uxac00uxba70}{%
\chapter{들어가며}\label{uxb4e4uxc5b4uxac00uxba70}}

\textbf{한국 알(R) 사용자회}는 디지털 불평등 해소와 통계 대중화를 오픈 통계 패키지 개발을 2021년부터 추진하였습니다.
더불어 설치형 오픈 통계 패키지를 신종화 님께서 John Fox 교수님이 개발한 \texttt{Rcmdr} 기반으로 한글화 및 문서화에 10년 넘게 기여해주셨습니다. 이에 \textbf{한국 알(R) 사용자회}는 신종화님의 \texttt{Rcmdr} 거인의 어깨위에 디지털 불평등 해소와 통계 대중화를 위해 한 걸음 더 나아가게 되었습니다. 특히 신종화님께서 기여하신 한글화 및 문서를 근간으로 더 많은 분들이 오픈 통계 패키지를 사용할 수 있도록 \texttt{bookdown}으로 내용을 정리하여 통계 대중화가 한층 앞당겨질 것으로 기대됩니다.

신종화님께서 왜 오픈 통계 패키지로 \texttt{Rcmdr}를 근간으로 해야 하는지 이유를 명쾌하게 다음과 같이 정리해 주셨습니다.

R에는 여러 개의 GUI 작업도구들이 있습니다. 모두 목적이 분명하고, 좋은 도구이며, 일부는 현재도 향상작업이 진행되고 있습니다. 그럼에도 불구하고 R Commander를 위한 블로그 작업을 진행하는 이유는 크게 두가지 입니다.

첫째, R Commander는 직관적으로 기존의 기초통계학 도구와 유사합니다. Command Line 에서 작업하는 것에 익숙하지 않은, 또 어려움을 겪고 있는 사용자들에게 기초통계학분야를 학습하고 활용하는데 도움을 주기 위하여 R Commander가 개발되었습니다. 개발자인 John Fox 교수는 이 목적과 관리방향을 분명히하고 있습니다. 중급이상의 R 사용자/ 고급통계 연구자들에게는 R Commander가 불필요할 수 있습니다.

둘째, 지난 10년동안 R Commander의 메뉴 한글화작업을 진행해왔으며, 현재도 유지관리를 하고 있습니다. (이 정보는 R Commander 안의 Help \textgreater{} About Rcmdr 에 있습니다) {[}Translations: Korean, Chel Hee Lee, Dae-Heung Jang, and Shin Jong-Hwa{]} 지난 10년 동안 개인적인 메모 차원에서 R Commander 사용 및 한글화 관련 블로그 포스트를 만들고 관리되어 왔고 \href{http://modernity.tistory.com}{블로그}에 전체 과정이 고스란히 남아있고 계속적으로 유지관리될 것입니다.

\begin{itemize}
\tightlist
\item
  신종화 \href{https://rcmdr.kr/}{Rcmdr : R Commander}
\item
  \href{https://cran.r-project.org/web/packages/Rcmdr/index.html}{CRAN Rcmdr 패키지 정보}
\item
  \href{https://socialsciences.mcmaster.ca/jfox/Misc/Rcmdr/}{개발자 John Fox 교수의 Rcmdr 소개}
\item
  \href{https://modernity.tistory.com/}{FOSSER\_Ricoop}
\end{itemize}

\hypertarget{install}{%
\chapter{설치}\label{install}}

  \bibliography{book.bib,packages.bib}

\end{document}
